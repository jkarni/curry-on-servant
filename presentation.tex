\documentclass{beamer}

\usepackage{verbatim}
\usepackage{tcolorbox}
\usepackage{lipsum}

\title{Servant}
\subtitle{A Type-Level DSL for Web APIs}
\author{Julian K. Arni}
\date{Curry On, Prague, July 6th 2015}
%%%%%%%%%%%%%%%%%%%%%%%%%%%%%%%%%%%%%%%%%%%%%%%%%%%%%%%%%%%%%%%%%%%%%%%%%%%%%%

\begin{document}
\frame{\titlepage}

\setbeamertemplate{footline}[frame number]
\begin{frame}
\frametitle{Outline}
\tableofcontents[]
\end{frame}

%%%%%%%%%%%%%%%%%%%%%%%%%%%%%%%%%%%%%%%%%%%%%%%%%%%%%%%%%%%%%%%%%%%%%%%%%%%%%%
\section{Rethinking Web Frameworks}

\begin{frame}
\frametitle{The Main Idea}
\pause
\begin{tcolorbox}
Web API descriptions should be central
\end{tcolorbox}
\pause
\begin{enumerate}
\item Accords with development practice \pause
\item Clarifies behaviour of programs \pause
\item Is a reusable component
\end{enumerate}
\end{frame}

\begin{frame}[fragile]
\frametitle{An Example}
\begin{tcolorbox}{Counter API}
\begin{verbatim}
GET  /       obtain the current value
POST /step   increment the current value
\end{verbatim}
\end{tcolorbox}
\pause
\begin{verbatim}
type GetCounter = Get '[JSON] CounterVal
type StepCounter = "step" :> Post '[] ()
type CounterAPI = GetCounter :<|> StepCounter
\end{verbatim}
\end{frame}

\begin{frame}[fragile]
\frametitle{An Example}
\begin{tcolorbox}{Counter API}
\begin{verbatim}
type GetCounter = Get '[JSON] CounterVal
type StepCounter = "step" :> Post '[] ()
type CounterAPI = GetCounter :<|> StepCounter
\end{verbatim}
\end{tcolorbox}
\pause
\begin{verbatim}
newtype CounterVal = CounterVal {getCounterVal :: Int}
  deriving (ToJSON)

counterAPI :: Proxy CounterAPI
counterAPI = Proxy
\end{verbatim}
\end{frame}

\begin{frame}[fragile]
\frametitle{Documentation}
\begin{verbatim}
counterDocs :: String
counterDocs = markdown $ docs counterAPI
\end{verbatim}
\end{frame}

\begin{frame}[fragile]
\begin{verbatim}
## GET /

#### Response:

- Status code 200
- Headers: []

- Supported content types are:

    - `application/json`

- Response body as below.

```javascript
42
```
\end{verbatim}
\end{frame}

\begin{frame}[fragile]
\frametitle{Client Functions}
\begin{verbatim}
getCounter' :<|> stepCounter'
 = client counterAPI ( BaseUrl Http "localhost" 8000 )
\end{verbatim}
\pause
\begin{verbatim}
getCounter' :: EitherT ServantError IO CounterVal
setCounter' :: EitherT ServantError IO ()
\end{verbatim}
\end{frame}

\begin{frame}[fragile]
\frametitle{Safe links}
\begin{verbatim}
safeLink counterAPI (Proxy :: Proxy StepCounter)
-- > "step"
safeLink counterAPI (Proxy :: "doesntexist" :> Get '[] ())
-- > type error
\end{verbatim}
\end{frame}

\begin{frame}[fragile]
\frametitle{Server}
\begin{verbatim}

getCounter :: TVar CounterVal -> Server GetCounter
getCounter ctr = liftIO $ readTVarIO ctr

stepCounter :: TVar CounterVal -> Server StepCounter
stepCounter ctr
  = liftIO $ atomatically $ modifyTVar ctr (+ 1)
\end{verbatim}
\pause
\begin{verbatim}
server :: TVar CounterVal -> Server CounterAPI
server ctr = getCounter ctr
       :<|>  stepCounter ctr
\end{verbatim}
\pause
\begin{verbatim}
main = do
  initCtr <- newTVar 0
  run 8000 (serve counterAPI server)
\end{verbatim}
\end{frame}

%%%%%%%%%%%%%%%%%%%%%%%%%%%%%%%%%%%%%%%%%%%%%%%%%%%%%%%%%%%%%%%%%%%%%%%%%%%%%%
\section{Type-level DSLs}

\begin{frame}[fragile]
\frametitle{A Simple DSL}

\begin{tcolorbox}
\begin{verbatim}
Add (Add One One) One
\end{verbatim}
\end{tcolorbox} \pause

\begin{verbatim}
data Expr = Add Expr Expr
          | One

eval :: Expr -> Int
eval One = 1
eval (Add x y) = eval x + eval y
\end{verbatim}
\end{frame}



\begin{frame}[fragile]
\frametitle{A Simple DSL}

\begin{tcolorbox}
\begin{verbatim}
Add (Add One One) One
\end{verbatim}
\end{tcolorbox}

\begin{verbatim}
class Eval x where
    eval :: Proxy x -> Int
\end{verbatim}
\pause
\begin{verbatim}
data One
data Add a b
instance Eval One where
    eval _ = 1
instance (Eval a, Eval b) => Eval (Add a b) where
    eval _ = eval (Proxy :: Proxy a)
          +  eval (Proxy :: Proxy b)
\end{verbatim}
\end{frame}


\begin{frame}
\frametitle{Type-level Advantages}
\begin{enumerate}
\item Extensible \pause
\item Right side of phase distinction
\end{enumerate}
\end{frame}


\begin{frame}[fragile]
\frametitle{A Fancier DSL} \pause

\begin{verbatim}
data Hole
\end{verbatim}
\pause
\begin{verbatim}
class Eval' a where
    type Value a r :: *
    eval' :: Proxy a -> (Int -> r) -> Value a r
\end{verbatim}
\pause
\begin{verbatim}
instance Eval' One where
    type Value One r = r
    eval' _ ret = ret 1
\end{verbatim}
\pause
\begin{verbatim}
instance (Eval' a, Eval' b) => Eval' (Add a b) where
    type Value (Add a b) r = Value a (Value b r)
    eval' _ ret = eval' (Proxy :: Proxy a) (\v1 ->
                  eval' (Proxy :: Proxy b) (\v2 ->
                  ret (v1 + v2)))
\end{verbatim}
\pause
\begin{verbatim}
instance Eval' Hole where
    type Value Hole r = Int -> r
    eval' _ ret n = ret n
\end{verbatim}
\end{frame}

\begin{frame}[fragile]
\frametitle{A Fancier DSL}
\begin{verbatim}
type Succ = Add One Hole
\end{verbatim}\pause
\begin{verbatim}
succ = eval' (Proxy :: Proxy Succ) -- :: (Int -> Int)
succ 5 -- > 6
\end{verbatim}
\end{frame}

\begin{frame}[fragile]
\frametitle{A Fancier DSL}
\begin{verbatim}
type Sum = Add Hole Hole
\end{verbatim}\pause
\begin{verbatim}
sum = eval' (Proxy :: Proxy Sum) -- :: (Int -> Int -> Int)
sum 5 10 -- > 15
\end{verbatim}
\end{frame}

%%%%%%%%%%%%%%%%%%%%%%%%%%%%%%%%%%%%%%%%%%%%%%%%%%%%%%%%%%%%%%%%%%%%%%%%%%%%%%
\section{Servant DSL}

\begin{frame}[fragile]
\frametitle{Servant Grammar}
\begin{verbatim}
api        ::=  api :<|> api
           |    item :> api
           |    method

item       ::=  symbol
           |    ReqBody      ctypes type
           |    Capture      symbol stype
           |    ...
\end{verbatim}
\end{frame}
\begin{frame}[fragile]
\begin{verbatim}
method     ::=  Get     ctypes type
           |    Put     ctypes type
           |    Post    ctypes type
           |    Delete  ctypes type
           |    Raw
           |    ...

type       ::=  <Haskell Types>

ctypes     ::=  '[ctype, ...]

ctype      ::=  PlainText
           |    JSON
           |    HTML
           |    ...
\end{verbatim}
\end{frame}

\begin{frame}[fragile]
\frametitle{HasServer}

\begin{verbatim}
class HasServer api where
  type Server api :: *
  route :: Proxy api -> Server api -> RoutingApplication
\end{verbatim}
\pause
\begin{verbatim}
type RoutingApplication  =
       Request
   ->  (RouteResult Response -> IO ResponseReceived)
   ->  IO ResponseReceived

data RouteResult a = NotMatched | Matched a
\end{verbatim}
\end{frame}

\begin{frame}[fragile]
\frametitle{HasServer}
\begin{verbatim}
data GetText t

instance Show t => HasServer (GetText t) where
  type Server (GetText t) = EitherT ServantErr IO t

  route  ::  Proxy (GetText t) -> Server (GetText t)
         ->  RoutingApplication
  route _ handler request respond
    |   pathIsEmpty request
    &&  requestMethod request == methodGet = accept
    |   otherwise = respond NotMatched
     where
     accept = do
       e <- runEitherT handler
       respond $ case e of
         Right t   ->  Matched $ responseLBS ok200
                         [("Content-Type", "text/plain")]
                         (fromString (show t))
         Left err  ->  Matched $ responseServantErr err
\end{verbatim}
\end{frame}

\begin{frame}[fragile]
\frametitle{HasServer}
\begin{verbatim}
data a :<|> b = a :<|> b
infixr 8 :<|>

instance  (HasServer api1, HasServer api2) =>
          HasServer (api1 :<|> api2) where

  type  Server (api1 :<|> api2) =
          Server api1 :<|> Server api2

  route _ (handler1 :<|> handler2) request respond =
    route (Proxy :: Proxy api1) handler1 request $ \ r ->
      case r of
        Matched result  ->  respond (Matched result)
        NotMatched      ->  route (Proxy :: Proxy api2)
                              handler2 request respond
\end{verbatim}
\end{frame}

\begin{frame}
\frametitle{References}
\begin{description}
\item[Type-level DSL] L\"{a}mmel, Ralf and Ostermann, Klaus,
\textit{Software Extension and Integration with Type Classes},
\item[Servant WGP paper]
\url{http://haskell-servant.github.io/posts/2015-05-25-servant-paper-wgp-2015.html}
\item[Servant website] \url{http://haskell-servant.github.io/}
\end{description}
\end{frame}

\end{document}
